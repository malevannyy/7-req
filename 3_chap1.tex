\chapter{Бизнес-требования}
\label{ch:chap1}


\section{Исходные данные}
\label{sec:asis}

Имеется производственное предприятие выпускающее строительную продукцию.
Продукция характеризуется следующими параметрами:
\begin{itemize}
    \item ширина,
    \item высота,
    \item толщина,
    \item масса,
\end{itemize}

Площадь операционной территории доставки продукции составляет примерно 1~млн.~км2.
Население операционной территории примерно 30~млн.~человек.

Предприятие оперирует в трех сегментах рынка:
\begin{itemize}
    \item розничном,
    \item оптовом,
    \item сетевом.
\end{itemize}

Продукция производится и доставляется в соответствии с методом “just-in-time”~\cite{jit}.

Ежедневно в 17:00 департамент продаж формирует закрытый список заказов (обычно свыше 1000 шт.).
Количество продукции в каждом заказе варьируется от единиц до сотен изделий,
таким образом суммарный поток изделий составляет несколько тысяч изделий в день.

К 18:00 дня должен быть готов логистический план.

Флот предприятия представлен различными типами транспортных средств (грузовых автомобилей) имеющих следующие параметры:
\begin{itemize}
    \item длина,
    \item ширина,
    \item грузоподъёмность,
    \item стоимость пробега (тонна километров),
    \item стоимость простоя (при разгрузке).
\end{itemize}

В соответствии с правилами~\cite{dr} продукция должна перевозиться на транспортных контейнерах которые имеют следующие параметры:
\begin{itemize}
    \item ширина,
    \item высота,
    \item глубина,
    \item количество сторон погрузки (1 или 2),
    \item вес.
\end{itemize}

Отдел логистики представлен следующими единицами:
\begin{itemize}
    \item старший логист,
    \item младший логист.
\end{itemize}

Ежедневный расчет логистического плана силами человеческой мысли возможен, но имеет ряд недостатков:
\begin{itemize}
    \item невозможно быстро понять является ли конкретный план созданный человеком достаточно оптимальным,
    \item качество составления плана зависит от состояния человека,
    \item человек склонен допускать ошибки.
\end{itemize}

Поэтому принято решение создать автоматизированную систему расчета логистического плана (АС).


\section{Возможности бизнеса}
\label{sec:possibilities}

Автоматизированная систему расчета логистического плана позволит бизнесу формализовать логистику доставки заказов,
что позволит получить экономию на масштабе, так как упорядоченность, учет и анализ прошлого опыта доставки заказов позволят
доставлять большее количество заказов в единицу времени, что при тех же плановых расходах на ведение логистики создаст дополнительную прибыль.

Отсутствие порядка создаёт значительные риски своевременной доставки продукции что отрицательно сказывается на имидже компании.

Расчет логистического плана силами автоматизированной системы позволит получить следующие преимущества:
\begin{itemize}
    \item прогнозируемое качество составления плана,
    \item снижение количества ошибок при составлении плана,
    \item повышение скорости расчёта плана,
    \item увеличение объёма планируемой к доставке продукции,
    \item снижение когнитивной нагрузки на персонал логистического отдела,
    \item снижение требований к персоналу логистического отдела,
    \item возможность сократить персонал логистического отдела,
\end{itemize}

Система должна интегрироваться с имеющимися ERM и CAD/CAM системами предприятия, что обеспечит бесшовнее движение информации внутри КИС предприятия.

Имеющиеся на рынке "коробочные" решения не покрывают всю полноту поставленной задачи.
Интеграция множества несовместимых "коробок" кране затруднена отсутствием в отрасли общепринятого формата обмена информацией.

Стоимость приобретения, внедрения множество коробок сопоставима со стоимостью разработки.

Неизбежные утечки информации о лучших практиках через поставщиков и внедренцев коробочных решений на узимом рынке приведут к потере конкурентного преимущества.
Напротив, разработка полностью контролируемой системы для нужд конкретного предприятия создаёт исключительные и трудно-копируемые конкурентные преимущества, что очень важно в конкурентной борьбе.


\section{Бизнес-цели и критерии успеха}
\label{sec:goals}

Бизнес целями проекта является:
\begin{itemize}
    \item повышение точности прогнозирования объёмов доставки не менее чем на 50\%, что позволит
    \item в 1.5 и более раз увеличить среднюю пропускную способность службы доставки,
    \item устранение человеческого фактора при расчете плана.
\end{itemize}


\section{Потребности клиента или рынка}
\label{sec:reqirements}

Потребности клиента, в данном случае - предприятия, состоят в следующем:
\begin{enumerate}
    \item Система должна вести учет логистических планов,
    \item Система должна иметь интеграцию с существующей ERP и CAD/CAM предприятия,
    \item Система должна позволять создавать и редактировать логистический план,
    \item Система должна понимать и использовать человеко-понятные адреса доставки,
    \item Система должна производить автоматический расчёт плана не более чем за 1 час,
    \item Водитель транспортного средства, получив свой фрагмент логистического плана, должен иметь полную картину плана предстоящей работы на день.
\end{enumerate}


\section{Бизнес-риски}
\label{sec:risks}

Автоматический расчет плана несет следующие риски:
\begin{itemize}
    \item зависимость предприятия от используемых алгоритмов и их реализации,
    \item зависимость предприятия от поставщика услуг геокодирования,
\end{itemize}

Данные риски можно парировать следующими путями:
\begin{itemize}
    \item в процессе расчёта, использовать как проверенные так и новые алгоритмы,
    \item АС должна иметь фитнес-функцию плана, позволяющую быстро сравнивать различные решения,
    \item использовать несколько поставщиков услуг геокодирования,
    \item кешировать результаты геокодирования.
\end{itemize}

Отсутствие АС несет следующие риски:
\begin{itemize}
    \item невозможность быстрого увеличения объемов планирования приводит к потере операционной прибыли ежедневно,
    \item ошибки при составлении логистического плана негативно влияют на бизнес-имидж предприятия в глазах потребителей,
    \item неоптимальное планирование ведет к излишнему расходу ГСМ автотранспортом и большему рабочему времени водителей, что увеличивает издержки.
\end{itemize}

\endinput
