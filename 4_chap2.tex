\chapter{Образ решения}
\label{ch:chap2}


\section{Положения об образе проекта}
\label{sec:vision}

Автоматизированная система расчёта логистического плана (АС) разрабатывается для специалистов отдела логистики производственного предприятия ``П'',
которым требуется иметь инструмент для автоматического расчета логистического плана и управления логистикой доставки продукции предприятия.

Эта АС является системой автоматизации логистики и одновременно системой поддержки принятия решений,
которая обеспечивает ежедневный расчет логистического плана и управление логистикой доставки продукции в целом.

В отличие от интеграции множества разрозненных несовместимых программных продуктов сторонних разработчиков,
наш продукт учитывает реальные потребности конкретного предприятия и создаст трудновоспроизводимые конкурентные преимущества для данного предприятия.


\section{Основные функции}
\label{sec:fun}

\begin{enumerate}[label=FUN-\arabic*]
    \item Система должна иметь список логистических планов по датам - Критический.
    \item Система должна иметь возможность открыть план на конкретную дату как в будущем так и в прошлом - Критический.
    \item Система должна давать возможность создавать план но не более одного на конкретную дату - Критический.
    \item Система должна получать заказы к доставке из ERP предприятия на конкретную дату, количество актов получения данных не ограничено - Критический.
    \item Система должна обеспечивать уникальность размещения заказов в логистическом плане - Критический.
    \item Система должна автоматически накладывать адреса доставки на маршрутную сеть дорог общего пользования - Критический.
    \item Система должна автоматически составлять логистический план доставки заказов - Критический.
    \item Система должна предоставлять возможность редактирования плана доставки заказов - Важный.
    \item Система должна иметь возможность вывода логистического плана в документ формата PDF который может быть распечатан - Критический.
    \item Система должна экспортировать план в ERP предприятия - Критический.
\end{enumerate}

\section{Предположения и зависимости}
\label{sec:dep}

Предположения:
\begin{enumerate}[label=ASM-\arabic*]
    \item Картографической информации предоставляемой провайдерами можно доверять в 90\% случаев,
    \item Геокодирование осуществляется с приемлемой точностью в 95\% случаев,
    \item Маршрутная сеть дорог общего пользования покрывает 95\% адресов доставки,
    \item Предприятие имеет потребный размер флота, а при недостатке имеет возможность заказать транспорт на стороне на любую дату доставки как минимум на дату формирования плана,
    \item Вычислительная платформ предприятие сможет предоставить дополнительные вычислительные мощности в течение 5 минут при выявлении недостаточности ресурсов в процессе выполнения планирования,
    \item Объем заказов единичного заказчика составляет не более 50\% общего потока заказов.
\end{enumerate}

Внешние зависимости:
\begin{enumerate}[label=DEP-\arabic*]
    \item Провайдеры геокодирования,
    \item Провайдеры картографической информации,
    \item Провайдеры информации сети дорог общего пользования,
    \item Строительные нормы и правила,
    \item Правила перевозки,
    \item Правила дорожного движения,
    \item Прочие нормативные акты данной юрисдикции.
\end{enumerate}

\endinput
