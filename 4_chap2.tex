\chapter{Образ решения}
\label{ch:chap2}


\section{Положения об образе проекта}
\label{sec:vision}

Автоматизированная система расчёта логистического плана (АС) разрабатывается для специалистов отдела логистики производственного предприятия "П",
которым требуется иметь инструмент для автоматического расчета логистического плана и управления логистикой доставки продукции предприятия.

Эта АС является системой автоматизации логистики и одновременно системой поддержки принятия решений,
которая обеспечивает ежедневный расчет логистического плана и управление логистикой доставки продукции в целом.

В отличие от интеграции множества разрозненных несовместимых программных продуктов сторонних разработчиков,
наш продукт учитывает реальные потребности конкретного предприятия и создаст трудновоспроизводимые конкурентные преимущества для данного предприятия.


\section{Основные функции}
\label{sec:fun}

\begin{enumerate}
    \item Система должна иметь список логистических планов по датам - Критический.
    \item Система должна иметь возможность открыть план на конкретную дату как в будущем так и в прошлом - Критический.
    \item Система должна давать возможность создавать план но не более одного на конкретную дату - Критический.
    \item Система должна получать заказы к доставке из ERP предприятия на конкретную дату, количество актов получения данных не ограничено - Критический.
    \item Система должна обеспечивать уникальность размещения заказов в логистическом плане - Критический.
    \item Система должна автоматически накладывать адреса доставки на маршрутную сеть дорог общего пользования - Критический.
    \item Система должна автоматически составлять логистический план доставки заказов - Критический.
    \item Система должна предоставлять возможность редактирования плана доставки заказов - Важный.
    \item Система должна иметь возможность вывода логистического плана в документ формата PDF который может быть распечатан - Критический.
    \item Система должна экспортировать план в ERP предприятия - Критический.
\end{enumerate}

\section{Предположения и зависимости}
\label{sec:dep}

Заинтересованные стороны:
\begin{enumerate}
    \item Владелец предприятия,
    \item Руководитель предприятия,
    \item Руководитель отдела продаж,
    \item Руководитель производственного отдела,
    \item Руководитель отдела логистики,
    \item Логистик,
    \item Клиент.
\end{enumerate}

Требования Владельца предприятия:
\begin{itemize}
    \item Снижение издержек,
    \item Повышение прибыли,
    \item Повышение масштабируемости бизнеса.
\end{itemize}

Требования Руководителя предприятия:
\begin{itemize}
    \item Повышение управляемости,
    \item Повышение предсказуемости результатов.
\end{itemize}

Требования Руководителя отдела продаж:
\begin{itemize}
    \item Список заказов на дату отдаётся в работу ежедневно не ранее 17:00,
    \item Срочные заказы должны приниматься в работу и после составления плана, но не позднее даты производства на дату доставки, (может привести к пересчету плана),
    \item Отмена заказа должна возможна, но не позднее до даты производства (может привести к пересчету плана),
    \item Возможность указывать временные окна доставки,
    \item Отчет по плановому времени доставки.
\item \end{itemize}

Требования Руководителя производственного отдела:
\begin{itemize}
    \item Логистический план должен быть готов ежедневно к 18:00,
    \item Изменения (пересчет) логистического плана не должен влиять на план производства сформированный на основе исходного логистического плана,
    \item Повышение предсказуемости планирования,
    \item Повышение точности планирования,
    \item Распределение иделий по транспортным контейнерам,
    \item Отчет по порядку погрузки изделий на транспортные контейнеры по-контейнерно.
\end{itemize}

Требования Руководителя отдела логистики:
\begin{itemize}
    \item Ретроспективный анализ исполнения логистических планов,
    \item Хранение планов за текущий и прошлый год.
\end{itemize}

Требования Логистика:
\begin{itemize}
    \item План составляется на конкретную дату,
    \item Автоматическая загрузка заказов на дату доставки,
    \item Ручное добавление заказа по номеру заказа, по клиенту,
    \item Удаление заказа из плана,
    \item Учет заказов не взятых в работу на дату доставки,
    \item Уникальность заказа в плане,
    \item Автоматическая привязка точки доставки заказа к сети дорог общего пользования,
    \item Отчет по плану в целом,
    \item Отчет по маршруту по-машино,
    \item Отчет по порядку погрузки транспортных контейнеров по-машинно,
    \item Отчет по порядку погрузки изделий на транспортные контейнеры по-контейнерно.
\end{itemize}

Требования Клиента:
\begin{itemize}
    \item Получение информации о процессе доставки заказа.
\end{itemize}

Внешние зависимости:
\begin{enumerate}
    \item Провайдеры геокодирования,
    \item Провайдеры картографической информации,
    \item Провайдеры информации о сети дорог общего пользования,
    \item Строительные нормы и правила,
    \item Правила перевозки,
    \item Правила дорожного движения,
    \item Прочие нормативные акты данной юрисдикции.
\end{enumerate}

\endinput
