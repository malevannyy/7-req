\chapter{Рамки и ограничения проекта}
\label{ch:chap3}


\section{Объем первой версии}
\label{sec:mvp}

Функции, которые необходимо реализовать в первой версии:
\begin{enumerate}[label=MVP-\arabic*]
    \item Список логистических планов по датам, не более одного плана на дату,
    \item Загрузка заказов из ERP,
    \item Редактирование адреса доставки,
    \item Геокодирование адресов заказов и привязка к сети дорог общего пользования,
    \item Составление логистического плана доставки заказов (план маршрутов ТС),
    \item Составление плана погрузки изделий (контейнерный план),
    \item Обеспечение уникальности заказа в плане,
    \item Экспорт логистического плана в ERP (порядок погрузки изделий на транспортные контейнеры),
    \item Отчёты: логистического план, маршрутный лист, погрузочный план.
\end{enumerate}


\section{Объем последующих версий}
\label{sec:next}
Функции, которые можно реализовать впоследствии:
\begin{enumerate}[label=NXT-\arabic*]
    \item Редактирование плана,
    \item Добавление срочных заказов,
    \item Удаление отмененных заказов,
    \item Ретроспективный анализ исполнения планов,
    \item On-line мониторинг исполнения плана текущего дня,
    \item Обеспечение доступа клиентов к информации о процессе доставки заказа.
\end{enumerate}


\section{Ограничения и исключения}
\label{sec:limit}
\begin{enumerate}[label=LMT-\arabic*]
    \item Система составляет логистический план для территории Северо-Западного федерального округа,
    \item Доставка по некоторым адресам невозможна вследствие отсутствия приемлемых дорог и подъездных путей, данные заказы планируются к доставке до ближайшей точке маршрутной сети дорог общего пользования,
    \item Адреса не включающие в себя достаточно информации для идентификации потребуют правки адреса, пример: 3-я улица Строителей, дом 25, кв 12.
\end{enumerate}

\endinput
